\chapter{Activity Diagrams And Use Case Diagrams}\label{ch:activity and use case-diagram}

\section{Use Case Diagram for Use Case UC-01}\label{uc-01}

\noindent\textbf{Linked to the use case:} \hyperlink{UC-01}{UC-01}
\begin{figure} [H]
\includegraphics[width=\textwidth]{png/use case diagram 1.png}
  \centering
  \caption{Use Case Diagram for UC-01}
  \vspace{-0.3cm}
\end{figure}




\section{Use Case Diagram for Use Case UC-02}\label{uc-02}

\noindent\textbf{Linked to the use case:} \hyperlink{UC-02}{UC-02}

\begin{figure} [H]
\includegraphics[width=\textwidth]{png/usecase diagram 2.png}
  \centering
  \caption{Use Case Diagram for UC-02}
  \vspace{-0.3cm}
\end{figure}


\newpage
\section{Use Case Diagram for Use Case UC-03}\label{uc-03}

\noindent\textbf{Linked to the use case:} \hyperlink{UC-03}{UC-03}

\begin{figure} [H]
\includegraphics[width=\textwidth]{png/use case diagram 3.png}
  \centering
  \caption{Use Case Diagram for UC-03}
  \vspace{-0.3cm}
\end{figure}

\newpage




\section{Activity Diagram For Use Case UC-01}\label{uc-01}

\noindent\textbf{Linked to the use case:} \hyperlink{UC-01}{UC-01}

\begin{figure} [H]
\includegraphics[width=\textwidth]{png/activity diagram 1.png}
  \centering
  \caption{Activity Diagram for UC-01}
  \vspace{-0.3cm}
\end{figure}




\section{Activity Diagram For Use Case UC-02}\label{uc-02}

\noindent\textbf{Linked to the use case:} \hyperlink{UC-02}{UC-02}


\begin{figure} [H]
\includegraphics[width=\textwidth]{png/activity diagram 2.png}
  \caption{Activity Diagram for UC-02}
  \vspace{-0.3cm}
\end{figure}

\newpage


\section{Activity Diagram For Use Case UC-03}\label{uc-03}

\noindent\textbf{Linked to the use case:} \hyperlink{UC-03}{UC-03}

\begin{figure} [H]
\includegraphics[width=\textwidth]{png/activity diagram 3.png}
  \centering
  \caption{Activity Diagram for UC-03}
  \vspace{-0.3cm}
\end{figure}


\section{Use Case Specifications For Use Case}







\definecolor{ucHeader}{HTML}{EEF2F7}
\definecolor{ucLabel}{HTML}{F7F9FC}
\newcolumntype{L}[1]{>{\RaggedRight\arraybackslash\bfseries}p{#1}}
\newcolumntype{Y}{>{\RaggedRight\arraybackslash}X}


\begin{table}[htbp]
\centering
\setlength{\tabcolsep}{6pt}
\renewcommand{\arraystretch}{1.2}
\begin{tabularx}{\linewidth}{L{0.30\linewidth} Y}
\toprule
\rowcolor{ucHeader}
\multicolumn{2}{l}{\textbf{Use case: Create UML Diagram and Export to EPS }}\\
\midrule
\rowcolor{ucLabel} Use case identifier & \hyperlink{UC-01}{UC-01} \\
\textbf{Brief description} & A user launches the UML tool, creates (or opens) a diagram, and exports the result to an EPS file. \\
\rowcolor{ucLabel} \textbf{Primary actors} & User \\
\textbf{Secondary actors} & File System  \\
\rowcolor{ucLabel} \textbf{Preconditions} &
\begin{enumerate}[leftmargin=*, itemsep=1pt, topsep=2pt]
  \item Application is available and starts successfully.
  \item User has permission to create and save files.
  \item Workspace is initialized.
\end{enumerate} \\
\textbf{Main flow} &
\begin{enumerate}[leftmargin=*, itemsep=1pt, topsep=2pt]
  \item User launches the tool.
  \item System loads the workspace and shows \emph{New} or \emph{Open}.
  \item User chooses \emph{New} diagram or load diagram.
  \item User adds/arranges blocks, connects them, and edits text.
  \item System validates UML notation.
  \item System creates a version snapshot.
  \item User chooses \emph{Export} $\rightarrow$ \emph{EPS}.
  \item System confirms success and shows the file location.
\end{enumerate} \\
\rowcolor{ucLabel} \textbf{Postconditions} &
\begin{enumerate}[leftmargin=*, itemsep=1pt, topsep=2pt]
  \item EPS file exists at the chosen location.
  \item Latest diagram state is saved (version snapshot updated).
\end{enumerate} \\
\textbf{Alternative flows} &
\textbf{Open existing} — In Step 3 the user selects \emph{Open}; the system loads a prior diagram, then continue at Step 4.\\[-2pt]
& \textbf{Validation issues} — In Step 5 the system highlights problems; the user fixes the diagram and re-validates, then proceed to Step 6.\\[-2pt]
& \textbf{Export failure} — In Step 8 rendering or write fails; the system shows an error and the user retries or cancels. \\
\rowcolor{ucLabel} \textbf{Related requirements} &
\hyperlink{SR-01}{SR-01}, \hyperlink{SR-02}{SR-02}, \hyperlink{SR-03}{SR-03}, \hyperlink{SR-04}{SR-04}, \hyperlink{SR-07}{SR-07}; \hyperlink{UR-01}{UR-01}, \hyperlink{UR-02}{UR-02}; \hyperlink{DR-03}{DR-03}, \hyperlink{DR-04}{DR-04}; \hyperlink{NFR-01}{NFR-01}, \hyperlink{NFR-03}{NFR-03} \\
\bottomrule
\end{tabularx}
\caption{Use Case Specifications For Use Case UC-01}
\end{table}






\begin{table}[htbp]
\centering
\setlength{\tabcolsep}{6pt}
\renewcommand{\arraystretch}{1.2}
\begin{tabularx}{\linewidth}{L{0.30\linewidth} Y}
\toprule
\rowcolor{ucHeader}
\multicolumn{2}{l}{\textbf{Use case: Create UML Diagram with Guided Help (UC-02)}}\\
\midrule
\rowcolor{ucLabel} Use case identifier & \hyperlink{UR-02}{UR-02} \\
\textbf{Brief description} &
A novice user creates a UML diagram using \emph{Guided Mode} with contextual hints and immediate help (docs/examples/assistant). \\
\rowcolor{ucLabel} \textbf{Primary actors} & New User \\
\textbf{Secondary actors} & Help/Docs Service, Assistant Chat, File System \\
\rowcolor{ucLabel} \textbf{Preconditions} &
\begin{enumerate}[leftmargin=*, itemsep=1pt, topsep=2pt]
  \item Application launches successfully; workspace is initialized.
  \item Help content is available (local cache or network).
\end{enumerate} \\
\textbf{Main flow} &
\begin{enumerate}[leftmargin=*, itemsep=1pt, topsep=2pt]
  \item User launches the tool; system offers \emph{Guided Mode}.
  \item System enables step-by-step hints and pins the Help panel.
  \item User chooses to start from a \emph{Template} or load diagram
  \item Guided hint: add and arrange blocks.
  \item Guided hint: connect blocks.
  \item Guided hint: edit text/labels.
  \item System validates UML notation (non-blocking issues shown inline).
  \item System saves a version snapshot.
  \item User can finish or continue editing (export optional, see UC-01).
\end{enumerate} \\
\rowcolor{ucLabel} \textbf{Postconditions} &
\begin{enumerate}[leftmargin=*, itemsep=1pt, topsep=2pt]
  \item A valid diagram exists and is saved.
  \item The user has received contextual guidance for key steps.
\end{enumerate} \\
\textbf{Alternative flows} &
\textbf{Switch to Standard} — At Step 1 the user selects Standard mode; proceed with UC-01 main flow.\\[-2pt]
& \textbf{Immediate Help} — At any step the user opens Help: search docs, view example snippet, or ask assistant; resume at the current step.\\[-2pt]
& \textbf{Validation Issues} — At Step 7 the system highlights problems; user applies fixes; re-validate, then continue to Step 8.\\[-2pt]
& \textbf{Help Unavailable} — If online help is unreachable, system falls back to local tips; user may retry later. \\
\rowcolor{ucLabel} \textbf{Related requirements} &
\hyperlink{SR-01}{SR-01}, \hyperlink{SR-02}{SR-02}, \hyperlink{SR-04}{SR-04}; \hyperlink{UR-04}{UR-04}; \hyperlink{DR-01}{DR-01}, \hyperlink{DR-03}{DR-03}. \\
\bottomrule
\end{tabularx}
\caption{Use Case UC-02: Guided creation with immediate help}
\end{table}


% UC-03 Use Case Table — Advanced editing / power-user features
\begin{table}[htbp]
\centering
\setlength{\tabcolsep}{6pt}
\renewcommand{\arraystretch}{1.2}
\begin{tabularx}{\linewidth}{L{0.30\linewidth} Y}
\toprule
\rowcolor{ucHeader}
\multicolumn{2}{l}{\textbf{Use case: Advanced Create \& Modify with Pro Features (UC-03)}}\\
\midrule
\rowcolor{ucLabel} Use case identifier & \hyperlink{UC-03}{UC-03} \\
\textbf{Brief description} &
An experienced user tailors a UML diagram with advanced controls (custom blocks, precise connections, styles, auto-layout, macros), validates it, and optionally exports (EPS/JPG/PNG). \\
\rowcolor{ucLabel} \textbf{Primary actors} & Experienced User \\
\textbf{Secondary actors} &
File System; Layout Engine; Style/Theme Manager; Library Manager (Custom Blocks); Versioning Service \\
\rowcolor{ucLabel} \textbf{Preconditions} &
\begin{enumerate}[leftmargin=*, itemsep=1pt, topsep=2pt]
  \item Application launches; workspace is initialized.
  \item User preferences (grid/snap, shortcuts) and libraries are available.
  \item User has permission to read/write project files.
\end{enumerate} \\
\textbf{Main flow} &
\begin{enumerate}[leftmargin=*, itemsep=1pt, topsep=2pt]
  \item User launches the tool; system loads workspace and preferences.
  \item User chooses \emph{New} or \emph{Open} diagram.
  \item \textbf{Advanced editing loop:}
        select block $\rightarrow$ edit properties /
        define custom block \& save to library / connect with constraints \& labels /
        align \& auto-layout / apply style or theme.
  \item System validates UML.
  \item System saves a version snapshot.
  \item (Optional) User compares with previous version and keeps/reverts.
  \item User chooses \emph{Export} $\rightarrow$ EPS (optionally JPG/PNG or batch).
  \item System renders and writes file(s), then confirms location.
\end{enumerate} \\
\rowcolor{ucLabel} \textbf{Postconditions} &
\begin{enumerate}[leftmargin=*, itemsep=1pt, topsep=2pt]
  \item Diagram with advanced edits is saved.
  \item Version history updated; any custom blocks/styles saved to library.
  \item Exported file(s) exist at the chosen location.
\end{enumerate} \\
\end{tabularx}
\caption{Use Case UC-03: Advanced editing and pro features}
\end{table}


% UC-03 Use Case Table — Advanced editing / power-user features
\begin{table}[htbp]
\centering
\setlength{\tabcolsep}{6pt}
\renewcommand{\arraystretch}{1.2}
\begin{tabularx}{\linewidth}{L{0.30\linewidth} Y}
\toprule
\rowcolor{ucHeader}
\multicolumn{2}{l}{\textbf{Use case: Advanced Create \& Modify with Pro Features (UC-03)}}\\
\midrule

\textbf{Alternative flows} &
\textbf{Missing assets} — When opening, referenced fonts/images not found; system prompts to locate/replace, then continue.\\[-2pt]
& \textbf{Validation violations} — Lint flags issues; system highlights and suggests fixes; user corrects and re-validates.\\[-2pt]
& \textbf{Export failure} — Render/write error; system shows message; user retries or cancels.\\[-2pt]
& \textbf{Revert changes} — User uses version compare to restore an earlier snapshot. \\
\rowcolor{ucLabel} \textbf{Related requirements} &
\hyperlink{SR-01}{SR-01}, \hyperlink{SR-02}{SR-02}, \hyperlink{SR-03}{SR-03}, \hyperlink{SR-04}{SR-04}, \hyperlink{SR-07}{SR-07}; \hyperlink{UR-01}{UR-01}, \hyperlink{UR-02}{UR-02}; \hyperlink{DR-03}{DR-03}, \hyperlink{DR-04}{DR-04};\hyperlink{NFR-01}{NFR-01}, \hyperlink{NFR-03}{NFR-03}. \\
\bottomrule
\end{tabularx}
\caption{Use Case UC-03: Advanced editing and pro features}

\clearpage



\section{Class Diagrams}




xxxx
\end{table}

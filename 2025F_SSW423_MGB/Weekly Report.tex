\chapter{Weekly Reports}
\index{Weekly Report} 
\label{Chapter::Weekly Report}

\section{Week Report 11 (11/13/2025)}
\hspace{2.5em}During this past week, we updated our document to have a version number included. We also added the ui design and updated acitivty diagrams. In the background, we have updated the project to allow for potential web application usage. We also managed to create an image of TikZiT that, for now, works locally on a web browser.

\hspace{1em} Next week, we will hope to have a presentation for the 3rd demo ready to go, as well as fix any mistakes regarding the things presented in said demo.

\textbf{What we have done:}
\begin{itemize}
    \item Added versioning
    \item Added Ui design
    \item Updated back end development
    \item Local web browser TikZiT
\end{itemize}

\textbf{What we are going to do:}
\begin{itemize}
    \item Prepare presentation for demo 3
    \item Fix anything that comes up during demo 3 that appears to need fixing.
\end{itemize}

\section{Week Report 10 (11/06/2025)}
\hspace{2.5em}During this past week, we updated our document to allow for dynamic updating with our requirements and use cases, as well as changing the labeling to better fit our style. We also got tikzit working with linux, and are getting a better understanding of the code

\hspace{1em} Next week, we will  further analyze the code for TikZit and continue working toward the prototype of the product. We are also planning on updating our diagrams to be more accurate. Creating an image of TikZiT that would work online would also be our goal throughought the next weeks.

\textbf{What we have done:}
\begin{itemize}
    \item Got tikzit working with linux
    \item Worked on tikzit code
    \item Updated requirement back end
\end{itemize}

\textbf{What we are going to do:}
\begin{itemize}
    \item Further work on TikZiT baseline
    \item Updating diagrams. 
    \item Creating a TikZiT image to launch on Digital Ocean Docker Droplet
\end{itemize}



\section{Week Report 9 (10/31/2025)}
\hspace{2.5em}During this past week, we focused on finalizing Chapter 6 and strengthening requirement traceability. We completed and inserted the use case and activity diagrams for UC-01, UC-02, and UC-03, and wrote full use-case specifications for the use cases. We also reviewed Tables in Chapter 2 to confirm UR/SR/DR/NFR coverage and tightened references across the document.

\hspace{1em} Next week, we will  further analyze the code for TikZit and start working toward the prototype of the product. We also planning on make the links for the use cases and requirements dynamic for easier use and link in the future. 

\textbf{What we have done:}
\begin{itemize}
    \item Added Use Case Diagrams and Activity Diagrams for all use cases.
    \item Wrote detailed Use Case Specifications for all use cases. 
    \item Updated user stories
\end{itemize}

\textbf{What we are going to do:}
\begin{itemize}
    \item Further work on TikZiT baseline
    \item In overleaf, make the links dynamic for all the use cases and requirements
    \item Working toward making the prototype/version 1.0
\end{itemize}



\section{Week Report 8 (10/24/2025)}
\hspace{2.5em} During this past week, we made significant progress in refining our project documentation and tool setup. We expanded the list of requirements and categorized them into system, user, domain, and non-functional requirements. These were then linked to their corresponding user stories and use cases to ensure proper traceability. We also updated our user stories and stakeholder lists to better reflect the current project scope. In addition, we successfully got TikZiT working on Windows, allowing all members to modify and run the source code consistently.

\hspace{1em} Next week, we will continue refining our baseline version of TikZiT and further develop our activity and use case diagrams. We will also begin working on detailed UML diagrams to support the project documentation and design process.

\textbf{What we have done:}
\begin{itemize}
    \item Added and categorized requirements (system, user, domain, non-functional)
    \item Linked requirements to user stories and use cases
    \item Updated user stories
    \item Updated stakeholders
    \item Successfully got TikZiT working on Windows
\end{itemize}

\textbf{What we are going to do:}
\begin{itemize}
    \item Further work on TikZiT baseline
    \item Continue developing activity and use case diagrams
    \item Begin working on UML diagrams for the project
\end{itemize}


\section{Week Report 7 (10/17/2025)}
\hspace{2.5em}  During this past week, we have continued to work on making sure that the source code for TikZiT can be installed and modified on a Windows machine. We are starting to understand what would be the best way to install the source code and how to modify it in a way that works. We also started to work on activity diagrams that would be of use for our project.

\hspace {1em} Next week, we will make sure that all of us have a steady way to work on the code from the source code. We will continue to look through the code and modify it for our needs. We will also make it so our activity diagrams are ready to be completed and put into our documentation.


What we have done:
\begin{itemize}
\item Set up source code for Windows
\item Work on activity diagrams
\item Looked through code and understand what we need to change

\end{itemize}
What we are going to do:

\begin{itemize}
\item Finish up activity diagrams
\item Get everyone up and working on the source code
\item Try to correctly build a working modified TikZiT app

\end{itemize}


\section{Week Report 6 (10/10/2025)}
\hspace{2.5em}  During this past week, we have added to the stakeholders list, worked on getting Tikzit to work on windows, and created a paper drawing of what the final code should result in for the player. We also started modifying and looking at the code, getting a small understanding of it.

\hspace {1em} Next week, we will continue to focus on getting the source code for TikZit on windows, further improve paper prototype, and further understanding of the code.


What we have done:
\begin{itemize}
\item Created a fully complete stakeholders list
\item Created basic prototype for finalized product
\item Worked directly with code for tikzit base

\end{itemize}
What we are going to do:

\begin{itemize}
\item Continue to work on getting windows started
\item Continue to flesh out paper prototype
\item Work on creating at least one feature 

\end{itemize}

\section{Week Report 5 (10/02/2025)}

\hspace {2.5em} During this past week, we had to work on a presentation, and in order to do that we ended up having to do a number of other things. This included working on a timeline, a clear list of roles and responsibilities, a development plan, coming up with our methods, and determining if an irb approval was needed. It was also spent turning everything into a presentation.


\hspace {1em} Next week, We will continue where we left off, since this week got derailed with catching up on things we missed and creating a presentation, so the to do list is the same.

What we have done:
\begin{itemize}
\item Came up with a list of responsibilities and who does what
\item Created a timeline and development plan, as well as determining more details on how we will do this
\item Creating a presentation for this information as well as the other information we had


\end{itemize}
What we are going to do:

\begin{itemize}
\item User story collection
\item Deeper understanding of Tikzit and Tinytex code
\item Basic prototyping
\end{itemize}


\section{Week Report 4 (09/25/2025)}

\hspace {2.5em}During this past week, we researched different tools and open source software to figure out how and where we want to start. Tikzit \cite{TikZiT} is a drawing tool that works directly with tikz, and makes it easier to port to latex. The reason we have decided to start with Tikzit for our basis is two-fold. Firstly, because of that link to latex it will make porting to overleaf more simplified, as overleaf is based entirely around latex. Secondly, it is a powerful drawing tool in general, and has a lot of what we are working for, so using it as a base means some of the features in our needs and want list are already fulfilled, such as nearly infinite zooming and being able to create new "blocks" while having some preset. We recompiled the source code of Tikzit and got it working, and played around with it, trying to get an understanding of fully how it works both in practice and in the code. A preview function exists inside of Tikzit, but to get it working, we need a pdflatex compiler that would compile our latex diagrams directly in the app. We found that using Tinytex \cite{TinyTex} works for our needs, but needs some tinkering to get it to fully function. We started to look into combining the two programs into one so that a user can get the diagrams and their previews all in one, especially since both of these programs are open source. We also worked on the presentation, fixing mistakes in the documentation, and also added some things like the glossary, and added some things to the bibliography. We also collected some user stories.


\hspace {1em} Next week, We will likely split into two camps, with half of our attention going to more user stories to get more defined ideas for what is needed, and the other is going to work with Tikzit/Tinytex more and start getting some coding done. 

What we have done:
\begin{itemize}
\item Decided on bases for project
\item Recompiled source code for Tikzit to get it working
\item Got Tikzit preview to work with the help of Tinytex
\item Fixed documentation and updated it where needed
\item Added glossary and bibliography
\item Worked on presentation
\item User story collection

\end{itemize}
What we are going to do:

\begin{itemize}
\item User story collection
\item Deeper understanding of Tikzit and Tinytex code
\item Basic prototyping
\end{itemize}

\section{Week Report 3 (09/18/2025)}

\hspace {2.5em}During this past week, we officially wrote out the use case and requirement tables, linking them with each other, adding a glossary to keep track of unique terms for our project and report, made the weekly reports an appendix, added the team declaration including key drivers and constraints, as well as adding a constraints table for future use. Also fixed some bugs with the documentation in order to make our document for readable and concise. We also took a closer look at the open source code and umlet. We decided against rushing into coding based off last week because we wanted to be more prepared for when we actually begin without rushing in without a plan.


\hspace {1em} Next week, the focus will be on user stories, how the conversion might work between a graphic, convert it to tkz, and convert THAT into latex and eps, look into open source programs that know how to draw or let the person draw using JAVA. We are also going into the constraints more as well. Finally, we are hoping to have a list of specific sites we can use as a refrence, either for user stories or how to begin the project.

What we have done:
\begin{itemize}
\item Use case and requirements table
\item Created a glossary
\item Bug fixing with documents 
\end{itemize}
What we are going to do:

\begin{itemize}
\item User story collection
\item Look more into prototyping
\item Refrences
\end{itemize}


\section{Week Report 2 (09/12/2025)}

\hspace {2.5em}During the past week our team concentrated on completing and refining the use case documentation for the MBG-EPS UML drawing tool. We carefully reviewed each user requirement, confirmed that every feature such as creating and exporting class, sequence, and activity diagrams is clearly tied to a corresponding use case, and ensured that the overall design remains practical for implementation.


\hspace {1em} Next week, we will turn our attention to planning the first coding sprint. We will begin coding the core application framework, set up the shared GitHub repository and testing environment, and start implementing basic UML element functionality. These next steps will mark our transition from documentation to active development and keep us on track with the project schedule.





\section{Week Report 1 (09/05/2025)}

\hspace {1em} This week we met up with our group members and created our Overleaf so that we can start documenting our Senior Project. We decided what we will be working on and how we will accomplish it. We looked into the project that allows a user to create a UML diagram that can then be exported into the EPS format. We are also thinking of making it a web based application.

Next we will start looking at the open source code of draw.io and umlet, as well as get a finalized plan so that by week 3 we can begin working in nearly full force. We will take a look at how the other applications work, what they do well, what they can do better, and how we might incorporate specific ideas. 
A list of action items.
What we have done:
\begin{itemize}
\item Create team
\item Created basic idea and plan
\item Created initial document 
\item Collecting ideas and resources to use as a reference.
\end{itemize}
What we are going to do:

\begin{itemize}
\item Download the source code
\item Create a list of specific features to include, or details from what we want from it
\item Create a (tentative) schedule for when we need certain features or milestones met
\item Create some diagrams for how the code should look and operate
\end{itemize}

One of the issues that we are experiencing is being able to use Overleaf with every member of our team. We are currently talking about it with the professor and will be able to get our Overleaf functioning in time to finish our assignments.


\cite{UMLet}

\index{Requirement} 
\chapter{Requirement}
\label{Requirement}


\section{Introduction}
Briefly state what is the problem you are trying to solve and perhaps a user story of how it will be used.

\section{Stakeholders}
\Large{See Chapter \ref{Stakeholders}}





\section{Key Concepts}
This can be a link to the Glossary chapter or table. I would recommend that you add a separate
Glossary chapter, for easier navigation, but you can use a Glossary table, too. Either is fine.




\section{All Requirements}
The requirements should be numbered and ideally in a long table, with columns for references
to use cases and priorities: Must, Could, Should, Would.




\section{User Requirements}
You may reference abstract statements written in natural language with accompanying informal
diagrams. You may also reference user stories or simple use case diagram(s).

\Large{See Chapter \ref{ch:activity-diagram} for use case diagram}


\section{System Requirements}
More detailed descriptions of the services and constraints from the perspective of the system to
meet the user requirements. Should be structured and precise. More detailed that describe the
user requirements and focus on the basic flow, alternative flow, exceptions, pre-conditions, and
post-conditions, and special requirements for each abstract user requirements. May use use-
case template for this.
System requirements should be numbered and traced back to the user requirements


\section{Non-functional}
Any important non-functional requirements for your project.




\section{Domain }
Business rules and regulations that impact what the system does.
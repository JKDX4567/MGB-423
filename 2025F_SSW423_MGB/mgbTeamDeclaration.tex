\index{Team Declaration} 
\chapter{Team Declaration}
\label{Team Declaration}


\section{Mission Statement}

\hspace {2.5em} Our project aims to allow students and programmers to easily create uml style \gls{diagram}s, and export them as an eps file ready to be used however the user sees fit. This will be done by creating an app that allows the user to drag on specific "\gls{block}s", connect, and edit them to display whatever the developer needs them to display for their project. The main focus of this project is to tackle the lackluster options that handle uml \gls{diagram} creation,  creating an easier to use system that handles dynamic needs, ensuring that users have a comfortable time creating uml \gls{diagram}s, being adaptable for those in the know, and allowing those who aren't knowledgeable to have a guide to help them practice this vital skill.


\section{Key Drivers}

\hspace {1em} For the key drivers, these are our motivation for making this project, and are the basis for many of our requirements. The key drivers are a hard to navigate system for most others, limited options, lack of eps conversion, and a desire to make it useful for those new to uml.

\begin{itemize}
\item Hard navigation systems. For most uml based applications they have features or mechanics that are not standard among other applications, and as such make them more difficult or annoying to use then is needed. For example, when using visual paradigm, when you copy something (such as a class), you are making a second literal copy, meaning changing one changes the other. This is antithetical to copying in most instances, since usually when one wants to copy, they want something closely the same bu not exactly. Basically, we feel these alternatives are not as intuitive or comfortable to use as can be, and how system should be able to adress this.
\item Limited options. While draw.io does allow for custom \gls{block}s, most other systems don't have an easy way to create your own \gls{block}s, let alone save them for later use. This means that those creating new systems entirely, or those who have specific things they want to adress, cannot do so as easily. Having the ability to create their own \gls{block}s allows for those with these specific needs to have them met and even reuse them.
\item Lack of eps conversion. This is rather straightforward, most of these systems don't allow you to create an eps file directly, and instead require you to convert from jpeg or png to eps, which is a hassle and leads to messy conversion issues. Our system will allow for a direct eps conversion, which will be useful for those using latex in particular.
\item Useful for those new to uml. When we were starting out, uml seemed a lot more complicated, and while we had the tools, we didnt know how to use them. Having a built in guide, with an example of what it does and how to show it, would have made our lives a lot easier, especially if it had code-to-uml examples. We want to include a guide in order to ensure new people are encouraged to use our system, and gives them a starting-off point.
\end{itemize}

\section{Key Constraints}

\hspace {1em} For the key constraints, these are the main things we have to take into account when creating our system. Essentially, they are the limitations, including our manpower, processing power, and our key focus.

\begin{itemize}
\item Manpower constraints. Our group is made up of three people. This is not necessarily a small amount, but its not a lot either. Most other groups are 4 or even 5, which means we have about 75-60 percent of the man-hours as other groups. This limits the scope we can attain with this project, and means we must be very careful with scope creep.
\item Processing power. Given that we are three students, we don't have access to the cutting edge tech that might allow for the app to have sloppy, inefficient code. Our code must be made more efficient than most, in order to ensure our application works for all of us. Doubly so given our group has both windows and mac, our application will have to be able to work with most operating systems. 
\item Key focus. Since this is an assigned project, specifically about uml and eps, our system priority will always be focused on that. This means that any decision, of it comes down to those two factors or something else, will always prioritize making it work in eps and have the correct uml \gls{diagram}s. Essentially, our focus is constrained to starting with that as the baseline, even if it goes against the natural course of the project, or the potential consumers wishes. 
\end{itemize}


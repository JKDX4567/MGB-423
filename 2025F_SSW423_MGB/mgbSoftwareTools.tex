\index{Software Tools} 
\chapter{Software Tools}
\label{Software Tools}


\section{TikZiT}
TikZiT is an open source GUI tool that allows a user to create their own diagrams that will be saves as a .tikz file and then either exported to Ovearleaf or the code is taken and pasted into Overleaf. This is the main tool that we'll be looking into incorporating into our project as it allows us to already draw some basic diagrams with .tikz functionality. What we'll have to do is modify it so that it fits out requirements. 

The following is a list of some of its features:
\begin{itemize}
    \item Custom Nodes and Lines
    \item Modify Code if Needed
    \item Simple Diagrams
    \item Latex Code in Diagrams
    \item Preview of Diagram (Latex Compiler Required)
    \item Potential automation with OverLeaf (file saves)

\end{itemize}

\noindent Example uses of TikZiT will be added later.

\section{TinyTex}
TinyTex is a latex compiler that is based on a bigger latex compiler TeX Live. The reason why we went with TinyTex is its small storage installation size, compared to the full TeX Live or other compilers. The reason why we need a Latex compiler is so that we can preview our diagrams while we are working on them in TikZiT. It might also be helpful in converting our diagrams into the EPS format.

The following is a list of some of its features:
\begin{itemize}
    \item No Real GUI
    \item Small Installation for a compiler (~350 MB)
    \item Helps with Latex compiling, especially for TikZiT
    \item Has built in Command Prompt
    \item Might have some issues with installation for TikZiT compiler to work, easy to fix, but not yet automated
    \item Might integrate directly into TikZiT for combined download

\end{itemize}
